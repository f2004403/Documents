%%%                                      2
%%%  Sri ramajayam

%%%%%%%%%%%%%%%%%%%%%%%%%%%%%%%%%%%%%%%%%%%%%%%%%%%%%%%%%%%%%%%%%%%%%%%%
%%%%%%%%%%%%%%%%%%%%%% Simple LaTeX CV Template %%%%%%%%%%%%%%%%%%%%%%%%
%%%%%%%%%%%%%%%%%%%%%%%%%%%%%%%%%%%%%%%%%%%%%%%%%%%%%%%%%%%%%%%%%%%%%%%%

%%%%%%%%%%%%%%%%%%%%%%%%%%%%%%%%%%%%%%%%%%%%%%%%%%%%%%%%%%%%%%%%%%%%%%%%
%% NOTE: If you find that it says                                     %%
%%                                                                    %%
%%                           1 of ??                                  %%
%%                                                                    %%
%% at the bottom of your first page, this means that the AUX file     %%
%% was not available when you ran LaTeX on this source. Simply RERUN  %%
%% LaTeX to get the ``??'' replaced with the number of the last page  %%
%% of the document. The AUX file will be generated on the first run   %%
%% of LaTeX and used on the second run to fill in all of the          %%
%% references.                                                        %%
%%%%%%%%%%%%%%%%%%%%%%%%%%%%%%%%%%%%%%%%%%%%%%%%%%%%%%%%%%%%%%%%%%%%%%%%

%%%%%%%%%%%%%%%%%%%%%%%%%%%% Document Setup %%%%%%%%%%%%%%%%%%%%%%%%%%%%

% Don't like 10pt? Try 11pt or 12pt
\documentclass[10pt]{article}

% This is a helpful package that puts math inside length specifications
\usepackage{calc}

% Layout: Puts the section titles on left side of page
\reversemarginpar

%
%         PAPER SIZE, PAGE NUMBER, AND DOCUMENT LAYOUT NOTES:
%
% The next \usepackage line changes the layout for CV style section
% headings as marginal notes. It also sets up the paper size as either
% letter or A4. By default, letter was used. If A4 paper is desired,
% comment out the letterpaper lines and uncomment the a4paper lines.
%
% As you can see, the margin widths and section title widths can be
% easily adjusted.
%
% ALSO: Notice that the includefoot option can be commented OUT in order
% to put the PAGE NUMBER *IN* the bottom margin. This will make the
% effective text area larger.
%
% IF YOU WISH TO REMOVE THE ``of LASTPAGE'' next to each page number,
% see the note about the +LP and -LP lines below. Comment out the +LP
% and uncomment the -LP.
%
% IF YOU WISH TO REMOVE PAGE NUMBERS, be sure that the includefoot line
% is uncommented and ALSO uncomment the \pagestyle{empty} a few lines
% below.
%

%% Use these lines for letter-sized paper
\usepackage[paper=letterpaper,
            includefoot, % Uncomment to put page number above margin
            marginparwidth=1in,     % Length of section titles
            marginparsep=.01in,       % Space between titles and text
            margin=0.18in,               % 1 inch margins
            includemp]{geometry}

%% Use these lines for A4-sized paper
%\usepackage[paper=a4paper,
%            %includefoot, % Uncomment to put page number above margin
%            marginparwidth=30.5mm,    % Length of section titles
%            marginparsep=1.5mm,       % Space between titles and text
%            margin=25mm,              % 25mm margins
%            includemp]{geometry}

%% More layout: Get rid of indenting throughout entire document
\setlength{\parindent}{0in}

%% This gives us fun enumeration environments. compactitem will be nice.
\usepackage{paralist}

%% Reference the last page in the page number
%
% NOTE: comment the +LP line and uncomment the -LP line to have page
%       numbers without the ``of ##'' last page reference)
%
% NOTE: uncomment the \pagestyle{empty} line to get rid of all page
%       numbers (make sure includefoot is commented out above)
%
\usepackage{fancyhdr,lastpage}
\pagestyle{fancy}
%\pagestyle{empty}      % Uncomment this to get rid of page numbers
\fancyhf{}\renewcommand{\headrulewidth}{0pt}
\fancyfootoffset{\marginparsep+\marginparwidth}
\newlength{\footpageshift}
\setlength{\footpageshift}
          {0.5\textwidth+0.5\marginparsep+0.5\marginparwidth-2in}
\lfoot{\hspace{\footpageshift}%
       \parbox{4in}{\, \hfill %
                    \arabic{page} of \protect\pageref*{LastPage} % +LP
%                    \arabic{page}                               % -LP
                    \hfill \,}}

% Finally, give us PDF bookmarks
\usepackage{color,hyperref}
\definecolor{darkblue}{rgb}{0.0,0.0,0.0}
\hypersetup{colorlinks,breaklinks,
            linkcolor=darkblue,urlcolor=darkblue,
            anchorcolor=darkblue,citecolor=darkblue}

%%%%%%%%%%%%%%%%%%%%%%%% End Document Setup %%%%%%%%%%%%%%%%%%%%%%%%%%%%


%%%%%%%%%%%%%%%%%%%%%%%%%%% Helper Commands %%%%%%%%%%%%%%%%%%%%%%%%%%%%

% The title (name) with a horizontal rule under it
%
% Usage: \makeheading{name}
%
% Place at top of document. It should be the first thing.
\newcommand{\makeheading}[1]%
        {\hspace*{-\marginparsep minus \marginparwidth}%
         \begin{minipage}[t]{\textwidth+\marginparwidth+\marginparsep}%
                {\large \bfseries #1}\\[-0.15\baselineskip]%
                 \rule{\columnwidth}{1pt}%
         \end{minipage}}

% The section headings
%
% Usage: \section{section name}
%
% Follow this section IMMEDIATELY with the first line of the section
% text. Do not put whitespace in between. That is, do this:
%
%       \section{My Information}
%       Here is my information.
%
% and NOT this:
%
%       \section{My Information}
%
%       Here is my information.
%
% Otherwise the top of the section header will not line up with the top
% of the section. Of course, using a single comment character (%) on
% empty lines allows for the function of the first example with the
% readability of the second example.
\renewcommand{\section}[2]%
        {\pagebreak[2]\vspace{0.8\baselineskip}%
         \phantomsection\addcontentsline{toc}{section}{#1}%
         \hspace{0in}%
         \marginpar{
         \raggedright \scshape #1}#2}

% An itemize-style list with lots of space between items
\newenvironment{outerlist}[1][\enskip\textbullet]%
        {\begin{itemize}[#1]}{\end{itemize}%
         \vspace{-0.6\baselineskip}}



% An environment IDENTICAL to outerlist that has better pre-list spacing
% when used as the first thing in a \section
\newenvironment{lonelist}[1][\enskip\textbullet]%
        {\vspace{-\baselineskip}\begin{list}{#1}{%
        \setlength{\partopsep}{0pt}%
        \setlength{\topsep}{0pt}}}
        {\end{list}\vspace{-.6\baselineskip}}

% An itemize-style list with little space between items
\newenvironment{innerlist}[1][\enskip\textbullet]%
        {\begin{compactitem}[#1]}{\end{compactitem}}

% To add some paragraph space between lines.
% This also tells LaTeX to preferably break a page on one of these gaps
% if there is a needed pagebreak nearby.
%\newcommand{\blankline}{\quad\pagebreak[1]}
\newcommand{\blankline}{\vspace{0.10cm}}
\newcommand{\spc}{\vspace{1mm}}
\newcommand{\spcless}{\vspace{1mm}}
\newcommand{\spcmore}{\vspace{0.2cm}}
%%%%%%%%%%%%%%%%%%%%%%%% End Helper Commands %%%%%%%%%%%%%%%%%%%%%%%%%%%

%%%%%%%%%%%%%%%%%%%%%%%%% Begin CV Document %%%%%%%%%%%%%%%%%%%%%%%%%%%%

\begin{document}
\makeheading{\textbf{C.V.Krishnakumar Iyer }}

\section{\textbf{Contact Information}}
%
% NOTE: Mind where the & separators and \\ breaks are in the following
%       table.
%
% ALSO: \rcollength is the width of the right column of the table
%       (adjust it to your liking; default is 1.85in).
%
\newlength{\rcollength}\setlength{\rcollength}{4in}%
%
\begin{tabular}[t]{@{}p{\textwidth-\rcollength}p{\rcollength}}
739 E El Camino Real Apt 112 & \textit{Email:}\href{mailto:cvkkumar@cs.stanford.edu}{cvkkumar@cs.stanford.edu}\\
Sunnyvale, CA-94087   &  Ph: 408-462-5745 
\end{tabular}

\section{\textbf{Interests}} Large Scale Data Mining and Analysis, Machine Learning, Information Retrieval and Software Engineering

\section{\textbf{Education}}
\href{http://www.stanford.edu/}{\textbf{Stanford University}},\hfill \textit{Sep '08 - Apr '10}
\begin{innerlist}
\item[] M.S. \href{http://www.cs.stanford.edu/}{Computer Science}\hfill{GPA : 3.86 / 4.0}
      \begin{innerlist}
        \item Recipient of the prestigious \emph{BPCL Scholarship} awarded by Bharat Petroleum Corporation Ltd. awarded to the elite few for graduate study in the US.
     
         \item \emph{Relevant Coursework:} \href{http://cs229.stanford.edu}{Machine Learning},  Data Mining \& E-Business, \href{http://www-stat.stanford.edu/~tibs/stat315a.html}{Elements of Statistical Learning}, \href{http://www.stanford.edu/class/msande239/}{Computational Advertising}, \href{http://cs276.stanford.edu}{ Information Retrieval \& Web Search},  \href{http://cs228.stanford.edu}{Probabilistic Graphical Models}, \href{http://cs345a.stanford.edu}{Data Mining}, \href{http://cs347.stanford.edu}{Transaction Processing and Distributed Databases}, \href{http://snap.stanford.edu/na09/}{Network Analysis}.
         %, Biomedical Systems Design(A+)
         %,  \href{http://www.stanford.edu/}{iPhone Application Programming}
         %, \href{http://cs245.stanford.edu}{Database Systems Principles}
        \end{innerlist}
\end{innerlist}

\blankline

\href{http://www.bits-goa.ac.in/}{\textbf{Birla Institute of Technology and Science, Pilani - Goa Campus}}\hfill \textit{Aug 04 - Jun 08}
\begin{innerlist}
\item[] M Sc.(Tech.) {Information Systems} \hfill{CGPA : 9.95/10.0}
        \begin{innerlist}
        \item Ranked \# 1 in Information Systems Department
        \item Recipient of the Silver Medal awarded by BITS to the second rank holder in the graduating class of 2008.
        \item \emph{Relevant Coursework : }Artificial Intelligence, Data Structures \& Algorithms, Software Engineering, Probability \& Statistics, Discrete Structures for Computer Science, Data Mining.
%%        \item Course Topper in 20 courses, including Data Structures and Algorithms, Symbolic Logic and Software Engineering through my undergraduate study.
        \end{innerlist}
\end{innerlist}

\section{\textbf{Experience}}
\href{http://www.apple.com}{\textbf{Apple Inc.}}\hfill \textit{Apr '10 to present}
%\begin{outerlist}
%\item[] Software Engineer - Maps Data Insights \hfill{May '13 - present}
%        \begin{innerlist}
%	\item TODO
%        \end{innerlist}
%\end{outerlist}
\begin{outerlist}
\item[] Software Engineer - Internet Services
	\begin{outerlist}
		\item[] Reputation and Indirect Fraud Detection on iTunes AppStore % \hfill{Sep '11 - May '13}
		\begin{innerlist}
			\item Responsible for the end-to-end development of the fraud solution to reputation fraud on iTunes AppStore as a part of a very small team - right from requirements gathering, conception, design, implementation and evaluation. 
			\item Experienced with data mining, network analysis and machine learning on massive scales.
			\item Implemented a scalable version of the \emph{Belief propagation algorithm} that works on k-partite graphs with billions nodes and tens of billions of edges.
			\item Responsible for communication of the working of system to different stakeholders.
			\item System currently in production for more than a year and has a very high hit-rate.
			\item Used Pig, Java MapReduce, Mahout and HBase on the Hadoop ecosystem.
		\end{innerlist}
	\end{outerlist}
	\begin{outerlist}
		\item[] Account Creation Fraud Detection on iTunes %\hfill{Jan '11 - May '12}
		\begin{innerlist}
			\item Designed and implemented a high performance and scalable subsystem of a system that risks account creations that works in the path of the transaction.
			\item Implemented a scalable stream mining algorithm that works on real time data streams.
			\item Implementation in Java.
		\end{innerlist}
	\end{outerlist}

	\begin{outerlist}
	\item[] Spam Detection on iTunes Ping comments %\hfill{Apr '10 - June '11}
	        \begin{innerlist}
	        \item Worked on machine learning techniques for spam detection on Ping. 
	        \end{innerlist}
	\end{outerlist}
\end{outerlist}

\begin{outerlist}
\item[]  Intern - Internet Services \hfill{June '09 to Mar '10}
\begin{innerlist}
        \item {Worked on automated sentiment analysis and opinion mining from Micro-blogs using a combination of sophisticated machine learning and data mining techniques.}
        \item {Part of a two-member team that was responsible for the entire project from its conception and design to implementation and production deployment. }
        \item {Project involved use and analysis of several state-of-the-art machine learning techniques for feature engineering, feature selection, skew handling in datasets, model comparisons for supervised learning, ensemble techniques for classifiers and evaluation of results.}
        \item {Used Java, libSVM, Weka.}
\end{innerlist}
\end{outerlist}

\blankline

\href{http://www.hpl.hp.com/india/}{\textbf{Hewlett Packard Labs-India}\hspace{1cm}\textit{Research Intern}\hfill \textit{Jan 08 to June 08}}\\
\emph{STAIR : System for Topical and Aggregated Information Retrieval.}
       \begin{compactitem}
        \item {Developed the architecture and the prototype of STAIR - an IR system that  applied of a combination of Collaborative analysis and Focused Crawling techniques on the web documents to provide personalized, consolidated information relevant to the used as an aggregated PDF document.}
        
        \item {Implemented using Java, on top of Lucene. The semantics information were obtained from WordNet.}
        \item Published as a \href{http://www.hpl.hp.com/techreports/2009/HPL-2009-51.pdf}{HP Labs Technical Report in 2009}.
       %\item \emph{Mentor : Dr. Krishnan Ramanathan}
       \end{compactitem}

\blankline

\href{http://www-csli.stanford.edu/}{\textbf{Center for Study of Language and Information, Stanford University}}\\
\textit{Graduate Research Assistant}\hfill \textit{Aug 08 to June 09}\\
 \emph{Cognitive Assistant that Learns and  Organizes ( CALO )}
       \begin{compactitem}
        \item Member of a team working on CALO, a system that extracts decisions from multi-party meetings to enable the effective handling of feedbacks.
        \item Also involved in the evaluation of new features to decision extraction process. 
        \item Implementation used Java and libSVM.
       %\item \emph{Mentor : Prof. Stanley Peters}
       \end{compactitem}

\blankline

\href{http://www.barc.ernet.in/}{\textbf{Bhabha Atomic Research Center, Trombay}}\hspace{0.1cm}\textit{Project Intern at DRHR}\hfill \textit{May 06 to July 06}\\
 \emph{Image Processing and Software Development for Simplifying Robot Trajectory Generation}
       \begin{compactitem}
        \item {Built a system  that extracts information of a continuous path from any arbitrary raw image using  graph-theoretic methods and provides input to the indigenous \emph{Sensor-cum-Manipulator}, a Parallel planar kinematic robot.}
          \item {Implemented using C, Matlab (for image processing) and VB (as a wrapper GUI).}
        %\item \emph{Mentors : Mr. Gaurav Bhutani and Dr. T.A. Dwarakanath}
        \end{compactitem}

%\pagebreak
\section{\textbf{Selected Projects}}
%

\textit{Role Discovery in Social Networks using Dirichilet Multinomial Regression Based Topic Modeling} 
\begin{compactitem}
\item{Implemented an unsupervised algorithm for identification of hierarchical roles in an organization using a combination of Social Network Analysis, Spectral Clustering and a variant of LDA (Latent Dirichilet Allocation).}
%\item{Implemented using Java and Matlab}.
\item \emph{Instructor: Prof. Daphne Koller, Stanford University}
\end{compactitem}
\spc

\textit{Recommendation Systems based on Delicious and Twitter} 
\begin{compactitem}
\item {Implemented a People Recommendation System on Twitter (in Python) by a combination of several algorithms, that included collaborative filtering and network analysis.}
\item{Designed and implemented an URL Recommendation system by analysis of tags from \emph{Delicious}.}
%\item \emph{Instructor: Prof. Andreas Weigend, Stanford University}
\end{compactitem}
\spc
\textit{URL Recommendation Based on Asymmetric Tag Similarity and Diffusion-Based Grouping}
\begin{compactitem}
\item {Implemented an URL Recommendation system by analysis of tag similarity using data from ShareThis and Delicious using MapReduce and Partition based Joins on top of the Aster Cluster.}
%\item \emph{Instructors : Prof. Jeff Ullman and Prof. Anand Rajaraman , Stanford University}
\end{compactitem}
\spc

\textit{Finding Answerers on Yahoo! Answers}
\begin{compactitem}
\item {Designed and implemented a system for selection of most appropriate answerers in \emph{Yahoo! Answers} based on textual, structural and other auxiliary information. The result could be used to determine the routing for new questions.}
%\item \emph{Instructor :  Prof.Jure Leskovic, Stanford University}
\end{compactitem}
\spc

\textit{Analysis of Text Based Classifiers}
\begin{compactitem}
\item {Implemented and analyzed the performance of different Na\"{\i}ve Bayes classifiers on the 20-Newsgroups dataset, using Java and Lucene.}
%\item \emph{Instructors :  Prof.Chris Manning and Prof. Prabhakar Raghavan , Stanford University}
\end{compactitem}
\spc

\textit{Comparison of Similarity Search Algorithms over Inverted Indexes}
\begin{compactitem}
\item {Implemented and analyzed the performance of commonly used indexing similarity search algorithms - Term-at-a-Time and Document-at-a-Time. Also optimized the algorithms  with efficient index compression.}
%\item \emph{Instructors :  Prof.Andrei Broder and Prof. Vanja Josifovski, Stanford University}
\end{compactitem}
\spc


%\textit{System for Man-in-the-Middle Attack}
%\begin{compactitem}
%\item {Implemented and analyzed a \textit{Man In The Middle} attack on a SSL/TLS Connection as a part of the course project for Cryptography.}
%%\item \emph{Instructors :  Prof.Dan Boneh, Stanford University}
%\end{compactitem}
%\spc

%\textit{Vocabulary Builder - An iPhone App Making Words Fun }
%\begin{compactitem}
%\item {Designed and developed an iPhone app to help students preparing for Vocabulary tests learn new %words, test themselves and receive instant graphical performance feedback.}
%\end{compactitem}
%\spc

%\textit{Paparazzi - A Flicker Photo Viewer}
%\begin{compactitem}
%\item { Developed an iPhone app to manage, view, geo-tag the photos from Flickr.  }
%\end{compactitem}
%\spc

\textit{RefMed - A Physician Referral and Review Service}
\begin{compactitem}
\item {Designed and developed a physician referral and review service that enables patients to review and rate the physicians and facilitates the physicians to recommend other doctors to their patients.}
%\item \emph{Instructor :  Prof. Amar Das, Stanford University}
\end{compactitem}
%\spc
%\textit{An Ontology-based Automatic Staging system for Cancer}
%\begin{compactitem}
%\item {Developed a Automatic Staging System for Breast Cancer over the NCI Thesaurus using SWRL, OWL and Prot\'{e}g\'{e}}.
%%\item \emph{Instructor :  Prof. Mark Musen, Stanford University}
%\end{compactitem}
%\spc
%\textit{Implementation Of a Search Engine for Personalized Information Retrieval by profiling of user data}
%\begin{compactitem}
%\item Designed and implemented  a prototype search engine that incorporates an additional dimension of Personalization through User Profiling for Enhanced relevancy.
%%\item \emph{Mentors : Dr. Bharat Deshpande and Mr. Mangesh Bedekar , BITS Pilani Goa Campus}
%\end{compactitem}
%\spc
\textit{Time Table Generator}
\begin{compactitem}
\item Implemented a system for automating the process of timetable generation for a University with its constraints, using a sub-optimal graph coloring approach for Constraint Satisfaction.
%\item{Algorithms implemented in C. UI implemented in VC++}
\end{compactitem}
\spc
%%\textit{Comprehending PageRank - its analysis and implementation }\hfill \textit{Aug 07 -- Dec 07}
%%\begin{compactitem}
%%\item Implementation and Analysis of basic link based ordering algorithms used in web search.
%%\item \emph{Mentor : Dr. J.V.Rao}
%%\end{compactitem}
%%\spc
%\textit{An Expert System for selection of Polymer Composite Systems }
%\begin{compactitem}
%\item Implemented a novel method for evaluation and ranking of constituent materials for composite products using \emph{TOPSIS} , an MADM (Multiple Attribute Decision Making) approach that ensures an optimum solution for the characteristics desired.
%%\item \emph{Mentor : Dr. Durai Prabhakaran, BITS Pilani Goa Campus}
%\end{compactitem}
%%\spc
%%\textit{Software Development for Management of Merit-cum-Need Scholarship}\hfill \textit{Jan 07 -- May 07}
%%\begin{compactitem}
%%\item Involved the Construction of Online, Web-Based and Database-Oriented System for managing the Merit cum Need Scholarships of the students of a University as a part of the team.
%%\item \emph{Mentor : Mr. Mangesh Bedekar}
%%\end{compactitem}

\section{\textbf{Publications}}
{C.V.Krishnakumar and Dr.Krishnan Ramanathan, \emph{STAIR : A System for Topical and Aggregated Information Retrieval}, Proceedings of  the  International Conference on Intelligent Human Computer Interaction (IHCI) 2009.}

\spcless
{R.T. Durai Prabhakaran, B.J.C. Babu, V.P. Agrawal, C.V. Krishna Kumar, \emph{A knowledge-based system for constituent material selection in polymer composite product design }-  Proceedings of ISRS-2006, International Symposium for Research Scholars, IIT-Madras.}

\section{\textbf{Skills}}
\emph{Languages}: Java, Pig, Python, C, SQL.\\
\emph{Frameworks}: Hadoop (Cloudera Certified Hadoop Developer), Spring, HBase, Mahout  \\
\emph{Tools and  Platforms}: Weka, \LaTeX{}, Basics of R and Teradata\\
\emph{Operating Systems}: Unix based Systems, Microsoft Windows.\\


%%%%%%%%%%%%%%%%%%%%%%%%%%%%%%%%%%%%%%%%%%%%%%%%%%%%%%%%%%%%%%%%%%%%%%%%%%%%%%%%%%%%%%%%%%%

\section{\textbf{Honors and Achievements}}
Consistent recipient of the \emph{Merit Scholarship} awarded by BITS Pilani to the top 10 students across the batch.

\spc

{Secured the First Prize at \emph{OpenSoft} - the software construction contest conducted as a part of QUARK-07, the national level technological fest at BITS-Pilani, Goa Campus.}

\spc

{Recipient of the Merit Certificate,awarded to the top 0.1\% of students, for proficiency in English in AISSCE from the CBSE, 2004.}


%%%%%%%%%%%%%%%%%%%%%%%%%% Reference %%%%%%%%%%%%%%%%%%%%%%%%%%%%%%%%%
%\section{\textbf{References}}
%Dr. Bharat Deshpande,\\
%Assistant Professor and Head of Department, \\
%BITS Pilani Goa Campus. \\
%E-mail : bmd@bits-goa.ac.in\\

%Mr. Vipul Pandey,\\
%Apple Inc, \\
%Cupertino, CA-95014 \\
%E-mail : vipul@apple.com\\


\end{document}
%%%%%%%%%%%%%%%%%%%%%%%%%% End CV Document %%%%%%%%%%%%%%%%%%%%%%%%%%%%%
